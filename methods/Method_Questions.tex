\documentclass[12pt]{article}
\renewcommand\abstractname{\textbf{ABSTRACT}}
%----------Packages----------
\usepackage{amsmath}
\usepackage{amssymb}
\usepackage{amsthm}
\usepackage{amsrefs}
\usepackage{dsfont}
\usepackage{mathrsfs}
\usepackage{stmaryrd}
\usepackage[all]{xy}
\usepackage[mathcal]{eucal}
\usepackage{verbatim}  %%includes comment environment
\usepackage{fullpage}  %%smaller margins
\usepackage{times}
\usepackage{multicol}
\usepackage{booktabs}
\usepackage{graphicx}
\usepackage{float}
%\usepackage{cite}
\usepackage{setspace}
%----------Commands----------
%%penalizes orphans
\clubpenalty=9999
\widowpenalty=9999
\providecommand{\abs}[1]{\lvert #1 \rvert}
\providecommand{\norm}[1]{\lVert #1 \rVert}
\usepackage{ amssymb }
\providecommand{\x}{\times}
\usepackage{sectsty}
\usepackage{lipsum}
\usepackage{titlesec}
\titleformat*{\section}{\normalsize\bfseries\scshape}
\titleformat*{\subsection}{\normalsize}
\titleformat*{\subsubsection}{\normalsize\bfseries\filcenter}
\titleformat*{\paragraph}{\normalsize\bfseries\filcenter}
\titleformat*{\subparagraph}{\normalsize\bfseries\filcenter}
\usepackage{indentfirst}
\providecommand{\ar}{\rightarrow}
\providecommand{\arr}{\longrightarrow}
%\hyphenpenalty  10000
%\exhyphenpenalty 10000
\usepackage[english]{babel}
\usepackage[utf8]{inputenc}
\usepackage{fancyhdr}
\usepackage[top=1in,bottom=1in,right=1in,left=1in,headheight=200pt]{geometry}
\pagestyle{fancy}
\lhead{$<$Group Name$>$}
\chead{}
\rhead{$<$Method Name$>$}
\cfoot{\thepage}
\titlespacing*{\section}{0pt}{0.75\baselineskip}{0.1\baselineskip}
\usepackage[labelfont=sc]{caption}
\captionsetup{labelfont=bf}
%\doublespacing

\begin{document}
\section{Description of the Method}
\subsection{Please provide a short (1-2 paragraph) summary of the general idea of the method.  What does it do and how?  If the method has been previously published, please provide a link to that work in addition to answering this and the following questions.}


\subsection{What is the method sensitive to? (e.g. asymmetric line shapes, certain periodicities, etc.)}


\subsection{Are there any known pros/cons of the method?  For instance, is there something in particular that sets this analysis apart from previous methods?}


\subsection{What does the method output and how is this used to mitigate contributions from photospheric velocities?}


\subsection{Other comments?}



\section{Data Requirements}
\subsection{What is the ideal data set for this method?  In considering future instrument design and observation planning, please comment specifically on properties such as the desired precision of the data, resolution, cadence of observations, total number of observations, time coverage of the complete data set, etc.}


\subsection{Are there any absolute requirements of the data that must be satisfied in order for this method to work?  Rough estimates welcome.}


\subsection{Other comments?}



\section{Applying the Method}
\subsection{What adjustments were required to implement this method on \texttt{EXPRES} data?  Were there any unexpected challenges?}


\subsection{How robust is the method to different input parameters, i.e. does running the method require a lot of tuning to be implemented optimally?}


\subsection{What metric does the method (or parameter tuner) use internally to decide if the method is performing better or worse?  (e.g. nightly RMS, specific likelihood function, etc.)} 


\subsection{Other comments?}



\section{Reflection on the Results}
\subsection{Did the method perform as expected?  If not, are there adjustments that could be made to the data or observing schedule to enhance the result?}


\subsection{Was the method able to run on all exposures?  If not, is there a known reason why it failed for some exposures?}


\subsection{If the method is GP based, please provide the kernel used and the best-fit hyper parameters with errors.  Otherwise, just write ``Not Applicable.''}


\subsection{Other comments?}


\section{General Comments}


\end{document}